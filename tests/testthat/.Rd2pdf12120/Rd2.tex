\documentclass[a4paper]{book}
\usepackage[times,inconsolata,hyper]{Rd}
\usepackage{makeidx}
\usepackage[utf8]{inputenc} % @SET ENCODING@
% \usepackage{graphicx} % @USE GRAPHICX@
\makeindex{}
\begin{document}
\chapter*{}
\begin{center}
{\textbf{\huge Package `Quartet'}}
\par\bigskip{\large \today}
\end{center}
\begin{description}
\raggedright{}
\inputencoding{utf8}
\item[Version]\AsIs{1.0.0}
\item[Date]\AsIs{2019-01-07}
\item[Title]\AsIs{Comparison of Phylogenetic Trees Using Quartet and Bipartition Measures}
\item[Description]\AsIs{Calculates the number of four-taxon subtrees consistent with a pair
of cladograms, calculating the symmetric quartet distance of Bandelt & Dress (1986),
Reconstructing the shape of a tree from observed dissimilarity data,
Advances in Applied Mathematics, 7, 309-343 <doi:10.1016/0196-8858(86)90038-2>,
and using the tqDist algorithm of Sand et al. (2014), tqDist: a library for
computing the quartet and triplet distances between binary or general trees,
Bioinformatics, 30, 2079–2080 <doi:10.1093/bioinformatics/btu157>
for pairs of bifurcating trees.}
\item[URL]\AsIs{}\url{https://github.com/ms609/Quartet}\AsIs{}
\item[BugReports]\AsIs{}\url{https://github.com/ms609/Quartet/issues}\AsIs{}
\item[Copyright]\AsIs{Incorporates code modified from tqDist <doi:10.1093/bioinformatics/btu157>.}
\item[License]\AsIs{GPL (>= 2)}
\item[Encoding]\AsIs{UTF-8}
\item[Language]\AsIs{en-GB}
\item[Depends]\AsIs{R (>= 3.4.0)}
\item[Imports]\AsIs{ape,
memoise,
phangorn,
Rcpp,
Rdpack,
Ternary (>= 1.0),
TreeSearch (>= 0.2.1)}
\item[Suggests]\AsIs{bookdown,
knitr,
rmarkdown,
testthat,
usethis}
\item[RdMacros]\AsIs{Rdpack}
\item[LinkingTo]\AsIs{Rcpp}
\item[LazyData]\AsIs{true}
\item[ByteCompile]\AsIs{true}
\item[VignetteBuilder]\AsIs{knitr}
\item[RoxygenNote]\AsIs{6.1.1}
\item[Roxygen]\AsIs{list(markdown = TRUE)}
\end{description}
\Rdcontents{\R{} topics documented:}
\inputencoding{utf8}
\HeaderA{AllQuartets}{List all quartets}{AllQuartets}
%
\begin{Description}\relax
Lists all choices of four taxa from a tree.

A more computationally efficient alternative to \code{\LinkA{combn}{combn}},
\code{AllQuartets} uses \code{\LinkA{memoise}{memoise}} to make repeated calls faster.
\end{Description}
%
\begin{Usage}
\begin{verbatim}
AllQuartets(n_tips)
\end{verbatim}
\end{Usage}
%
\begin{Arguments}
\begin{ldescription}
\item[\code{n\_tips}] Integer, specifying the number of tips in a tree.
\end{ldescription}
\end{Arguments}
%
\begin{Value}
Returns a list of length \code{choose(n\_tips, 4)}, with each entry
corresponding to a unique selection of four different integers less than
or equal to \code{n\_tips}
\end{Value}
%
\begin{Author}\relax
Martin R. Smith
\end{Author}
%
\begin{SeeAlso}\relax
\code{\LinkA{combn}{combn}}
\end{SeeAlso}
%
\begin{Examples}
\begin{ExampleCode}
 n_tips <- 6
 AllQuartets(n_tips)
 
 combn(n_tips, 4) # Provides the same information, but for large 
                  # values of n_tips is significantly slower.

\end{ExampleCode}
\end{Examples}
\inputencoding{utf8}
\HeaderA{CompareQuartets}{Compare quartet states by explicit enumeration}{CompareQuartets}
%
\begin{Description}\relax
Uses explicit enumeration to compare two lists of quartet states,
detailing how many are identical and how many are unresolved.
For most purposes, the faster function \LinkA{QuartetStatus}{QuartetStatus} will be preferable.
\end{Description}
%
\begin{Usage}
\begin{verbatim}
CompareQuartets(x, cf)
\end{verbatim}
\end{Usage}
%
\begin{Arguments}
\begin{ldescription}
\item[\code{x, cf}] List of quartet states, perhaps generated by
\code{\LinkA{QuartetStates}{QuartetStates}}.
\end{ldescription}
\end{Arguments}
%
\begin{Value}
Returns an array of seven numeric elements, corresponding
\begin{description}

\item[N] The total number of quartet \emph{statements} for two trees of \emph{n} tips,
i.e. 2 \emph{Q}.
\item[Q] The total number of quartets for \emph{n} tips.
\item[s] The number of quartets that are resolved identically in both trees.
\item[d] The number of quartets that are resolved differently in each tree.
\item[r1] The number of quartets that are resolved in tree 1, but not in tree 2.
\item[r2] The number of quartets that are resolved in tree 2, but not in tree 1.
\item[u] The number of quartets that are unresolved in both trees.

\end{description}

\end{Value}
%
\begin{Author}\relax
Martin R. Smith
\end{Author}
%
\begin{References}\relax

Estabrook GF, McMorris FR, Meacham CA (1985).
``Comparison of undirected phylogenetic trees based on subtrees of four evolutionary units.''
\emph{Systematic Zoology}, \bold{34}(2), 193--200.
doi:\nobreakspace{}\Rhref{http://doi.org/10.2307/2413326}{10.2307\slash{}2413326}.

\end{References}
%
\begin{SeeAlso}\relax
\code{\LinkA{QuartetStatus}{QuartetStatus}}, generates this output from a list of
trees.
\end{SeeAlso}
%
\begin{Examples}
\begin{ExampleCode}
  n_tip <- 6
  trees <- list(ape::rtree(n_tip, tip.label=seq_len(n_tip), br=NULL),
                ape::rtree(n_tip, tip.label=seq_len(n_tip), br=NULL))
  splits <- lapply(trees, TreeSearch::Tree2Splits)
  quartets <- QuartetStates(splits)
  CompareQuartets(quartets[[1]], quartets[[2]])

\end{ExampleCode}
\end{Examples}
\inputencoding{utf8}
\HeaderA{Distances}{Triplet and quartet distances with tqDist}{Distances}
\aliasA{AllPairsQuartetAgreement}{Distances}{AllPairsQuartetAgreement}
\aliasA{AllPairsQuartetDistance}{Distances}{AllPairsQuartetDistance}
\aliasA{AllPairsTripletDistance}{Distances}{AllPairsTripletDistance}
\aliasA{OneToManyQuartetAgreement}{Distances}{OneToManyQuartetAgreement}
\aliasA{PairsQuartetDistance}{Distances}{PairsQuartetDistance}
\aliasA{PairsTripletDistance}{Distances}{PairsTripletDistance}
\aliasA{QuartetAgreement}{Distances}{QuartetAgreement}
\aliasA{QuartetDistance}{Distances}{QuartetDistance}
\aliasA{TripletDistance}{Distances}{TripletDistance}
%
\begin{Description}\relax
Functions to calculate triplet and quartet distances between pairs of trees.
\end{Description}
%
\begin{Usage}
\begin{verbatim}
QuartetDistance(file1, file2)

QuartetAgreement(file1, file2)

PairsQuartetDistance(file1, file2)

OneToManyQuartetAgreement(file1, file2)

AllPairsQuartetDistance(file)

AllPairsQuartetAgreement(file)

TripletDistance(file1, file2)

PairsTripletDistance(file1, file2)

AllPairsTripletDistance(file)
\end{verbatim}
\end{Usage}
%
\begin{Arguments}
\begin{ldescription}
\item[\code{file, file1, file2}] Paths to files containing a tree or trees in Newick format.
\end{ldescription}
\end{Arguments}
%
\begin{Value}
\code{Distance} functions return the distance between the requested trees.

\code{Agreement} functions return the number of triplets or quartets that are:
\begin{itemize}

\item \code{A}, resolved in the same fashion in both trees;
\item \code{E}, unresolved in both trees.

\end{itemize}


Comparing a tree against itself yields the totals (\code{A+B+C}) and (\code{D+E})
referred to by Brodal \emph{et al}. (2013) and Holt \emph{et al}. (2014).
\end{Value}
%
\begin{Section}{Functions}
\begin{itemize}

\item \code{QuartetDistance}: Returns the quartet distance between the tree.
in \code{file1} and the tree in \code{file2}.

\item \code{QuartetAgreement}: Returns a vector of length two, listing [1]
the number of resolved quartets that agree (\code{A});
[2] the number of quartets that are unresolved in both trees (\code{E}).
See Brodal et al. (2013).

\item \code{PairsQuartetDistance}: Quartet distance between the tree on each line of \code{file1}
and the tree on the corresponding line of \code{file2}.

\item \code{OneToManyQuartetAgreement}: Quartet distance between the tree in
\code{file1} and the tree on each line of \code{file2}.

\item \code{AllPairsQuartetDistance}: Quartet distance between each tree listed in \code{file} and
each other tree therein.

\item \code{AllPairsQuartetAgreement}: Quartet status for each pair of trees in \code{file}.

\item \code{TripletDistance}: Triplet distance between the single tree given
in each file.

\item \code{PairsTripletDistance}: Triplet distance between the tree on each line of \code{file1}
and the tree on the corresponding line of \code{file2}.

\item \code{AllPairsTripletDistance}: Triplet distance between each tree listed in \code{file} and
each other tree therein.

\end{itemize}
\end{Section}
%
\begin{Author}\relax
\begin{itemize}

\item Algorithms: Brodal \emph{et al.} (2013); Holt \emph{et al.} (2014)
\item C implementation: Sand \emph{et al.} (2014) (modified for portability by MRS).
\item R interface: Martin R. Smith

\end{itemize}

\end{Author}
%
\begin{References}\relax

Brodal GS, Fagerberg R, Mailund T, Pedersen CNS, Sand A (2013).
``Efficient algorithms for computing the triplet and quartet distance between trees of arbitrary degree.''
\emph{SODA '13 Proceedings of the twenty-fourth annual ACM-SIAM symposium on Discrete algorithms}, 1814--1832.
doi:\nobreakspace{}\Rhref{http://doi.org/10.1137/1.9781611973105.130}{10.1137\slash{}1.9781611973105.130}.

Holt MK, Johansen J, Brodal GS (2014).
``On the scalability of computing triplet and quartet distances.''
In \emph{Proceedings of 16th Workshop on Algorithm Engineering and Experiments (ALENEX) Portland, Oregon, USA}.

Sand A, Holt MK, Johansen J, Brodal GS, Mailund T, Pedersen CNS (2014).
``tqDist: a library for computing the quartet and triplet distances between binary or general trees.''
\emph{Bioinformatics}, \bold{30}(14), 2079--2080.
ISSN 1460-2059, doi:\nobreakspace{}\Rhref{http://doi.org/10.1093/bioinformatics/btu157}{10.1093\slash{}bioinformatics\slash{}btu157}.

\end{References}
\inputencoding{utf8}
\HeaderA{PlotQuartet}{Plot quartet on a tree topology}{PlotQuartet}
%
\begin{Description}\relax
Draws a tree, highlighting the members of a specified quartet in colour.
\end{Description}
%
\begin{Usage}
\begin{verbatim}
PlotQuartet(tree, quartet, overwritePar = TRUE, caption = TRUE, ...)
\end{verbatim}
\end{Usage}
%
\begin{Arguments}
\begin{ldescription}
\item[\code{tree}] A tree of class \code{phylo}, or a list of such trees.

\item[\code{quartet}] A vector of four integers, corresponding to numbered tips on
the tree; or a character vector specifying the labels of four
tips.

\item[\code{overwritePar}] Logical specifying whether to use existing
\code{\LinkA{par}{par}} \code{mfrow} and \code{mar} parameters
(\code{FALSE}),
or to plot trees side-by-side in a new graphical device (\code{TRUE}).

\item[\code{caption}] Logical specifying whether to annotate each plot to specify
whether the quartet selected is in the same or a different state to the
reference tree.

\item[\code{...}] Additional parameters to send to \code{\LinkA{plot}{plot}}
\end{ldescription}
\end{Arguments}
%
\begin{Value}
Returns \code{invisible()}, having plotted a tree in which the first two members
of \code{quartet} are highlighted in orange, and the second two highlighted in
blue.
\end{Value}
%
\begin{Author}\relax
Martin R. Smith
\end{Author}
%
\begin{Examples}
\begin{ExampleCode}
  data('sq_trees')
  
  par(mfrow=c(3, 5), mar=rep(0.5, 4))
  PlotQuartet(sq_trees, c(2, 5, 3, 8), overwritePar = FALSE)

\end{ExampleCode}
\end{Examples}
\inputencoding{utf8}
\HeaderA{QuartetPoints}{Plot tree differences on ternary plots}{QuartetPoints}
\aliasA{BipartitionPoints}{QuartetPoints}{BipartitionPoints}
\aliasA{SplitPoints}{QuartetPoints}{SplitPoints}
%
\begin{Description}\relax
Generate points to depict tree difference (in terms of resolution
and accuracy) on a ternary plot.
\end{Description}
%
\begin{Usage}
\begin{verbatim}
QuartetPoints(trees, cf = trees[[1]])

SplitPoints(trees, cf = trees[[1]])

BipartitionPoints(trees, cf = trees[[1]])
\end{verbatim}
\end{Usage}
%
\begin{Arguments}
\begin{ldescription}
\item[\code{trees}] A list of trees of class \code{\LinkA{phylo}{phylo}}, with identically-labelled tips.

\item[\code{cf}] Comparison tree of class \code{\LinkA{phylo}{phylo}}.  If unspecified,
each tree is compared to the first tree in \code{trees}.
\end{ldescription}
\end{Arguments}
%
\begin{Details}\relax
The ternary plot will depict the number of quartets or bipartitions that are:
\begin{itemize}

\item resolved in the reference tree (\code{cf}), but neither present nor contradicted
in each comparison tree (\code{trees});
\item resolved differently in the reference and the comparison tree;
\item resolved in the same manner in the reference and comparison trees.

\end{itemize}


If the reference tree (\code{cf}) is taken to represent the best possible knowledge
of the 'true' topology, then polytomies in the reference tree represent
uncertainty.  If a tree in \code{trees} resolves relationships within this
polytomy, it is not possible to establish (based only on the reference tree)
whether this resolution is correct or erroneous.  As such, extra resolution
in \code{trees} that is neither corroborated nor contradicted by \code{cf} is ignored.
\end{Details}
%
\begin{Value}
A data frame listing the ternary coordinates of trees, based on the
amount of information that they have in common with the comparison
tree (which defaults to the first member of the list, if unspecified).
\end{Value}
%
\begin{Section}{Functions}
\begin{itemize}

\item \code{SplitPoints}: Uses partitions instead of quartets to calculate
tree distances.

\end{itemize}
\end{Section}
%
\begin{Author}\relax
Martin R. Smith
\end{Author}
%
\begin{References}\relax
Smith MR (in review).
``Bayesian and parsimony approaches reconstruct informative trees from simulated morphological datasets.''
\emph{Biology Letters; preprint at BioRxiv}.
doi:\nobreakspace{}\Rhref{http://doi.org/10.1101/227942}{10.1101\slash{}227942}.
\end{References}
%
\begin{Examples}
\begin{ExampleCode}
{
  library('Ternary')
  data('sq_trees')
  
  TernaryPlot(alab='Unresolved', blab='Contradicted', clab='Consistent', point='right')
  TernaryLines(list(c(0, 2/3, 1/3), c(1, 0, 0)), col='red', lty='dotted')
  TernaryText(QuartetPoints(sq_trees, cf=sq_trees$collapse_one), 1:15, 
    col=Ternary::cbPalette8[2], cex=0.8)
  TernaryText(SplitPoints(sq_trees, cf=sq_trees$collapse_one), 1:15, 
    col=Ternary::cbPalette8[3], cex=0.8)
  legend('bottomright', c("Quartets", "Bipartitions"), bty='n', pch=1, cex=0.8,
    col=Ternary::cbPalette8[2:3])
  
}

\end{ExampleCode}
\end{Examples}
\inputencoding{utf8}
\HeaderA{QuartetState}{Quartet State(s)}{QuartetState}
\aliasA{QuartetStates}{QuartetState}{QuartetStates}
%
\begin{Description}\relax
Report the status of the specified quartet(s).
\end{Description}
%
\begin{Usage}
\begin{verbatim}
QuartetState(tips, bips)

QuartetStates(splits)
\end{verbatim}
\end{Usage}
%
\begin{Arguments}
\begin{ldescription}
\item[\code{tips}] A four-element array listing a quartet of tips, either by their
number (if class \code{numeric}) or their name (if class \code{character}).

\item[\code{bips}] Bipartitions to evaluate.

\item[\code{splits}] A list of bipartition splits, perhaps generated using
\code{\LinkA{Tree2Splits}{Tree2Splits}}, with row names corresponding
to taxon labels.
If a tree or list of trees (of class \code{phylo}) is sent instead,
it will be silently converted to its constituent splits.
\end{ldescription}
\end{Arguments}
%
\begin{Details}\relax
One of the three possible four-taxon trees will be consistent with any set of bipartitions
generated from a fully resolved tree.  If the taxa are numbered 1 to 4, this tree can be
identified by naming the tip most closely related to taxon 1.
If a set of bipartitions is generated from a tree that contains polytomies, it is possible
that all three four-taxon trees are consistent with the set of bipartitions.
\end{Details}
%
\begin{Value}
Returns \code{0} if the relationships of the four taxa are not constrained by the provided
bipartitions, or the index of the closest relative to \code{tips[1]}, otherwise.
\end{Value}
%
\begin{Section}{Functions}
\begin{itemize}

\item \code{QuartetStates}: A convenience wrapper that need only be provided
with a tree or a list of splits

\end{itemize}
\end{Section}
%
\begin{Author}\relax
Martin R. Smith
\end{Author}
%
\begin{References}\relax
Estabrook GF, McMorris FR, Meacham CA (1985).
``Comparison of undirected phylogenetic trees based on subtrees of four evolutionary units.''
\emph{Systematic Zoology}, \bold{34}(2), 193--200.
doi:\nobreakspace{}\Rhref{http://doi.org/10.2307/2413326}{10.2307\slash{}2413326}.
\end{References}
%
\begin{SeeAlso}\relax
\code{\LinkA{CompareQuartets}{CompareQuartets}}, used to compare quartet states between
trees.
\end{SeeAlso}
%
\begin{Examples}
\begin{ExampleCode}
{
  n_tip <- 6
  trees <- list(ape::rtree(n_tip, tip.label=seq_len(n_tip), br=NULL),
                ape::rtree(n_tip, tip.label=seq_len(n_tip), br=NULL))
  splits <- lapply(trees, TreeSearch::Tree2Splits)
  QuartetState(c(1, 3, 4, 6), splits[[2]])  
  QuartetState(1:4, splits[[1]]) == QuartetState(1:4, splits[[2]])
  vapply(AllQuartets(n_tip), QuartetState, bips=splits[[1]], double(1))
}

\end{ExampleCode}
\end{Examples}
\inputencoding{utf8}
\HeaderA{ResolvedQuartets}{Count Resolved Quartets}{ResolvedQuartets}
\aliasA{ResolvedTriplets}{ResolvedQuartets}{ResolvedTriplets}
%
\begin{Description}\relax
Counts how many quartets are resolved or unresolved in a given tree,
following Brodal \emph{et al.} 2013.
\end{Description}
%
\begin{Usage}
\begin{verbatim}
ResolvedQuartets(tree, countTriplets = FALSE)

ResolvedTriplets(tree)
\end{verbatim}
\end{Usage}
%
\begin{Arguments}
\begin{ldescription}
\item[\code{tree}] A tree of class \code{\LinkA{phylo}{phylo}}.

\item[\code{countTriplets}] Logical; if \code{TRUE}, the function will return the number
of triplets instead of the number of quartets
\end{ldescription}
\end{Arguments}
%
\begin{Value}
A vector of length two, listing the number of quartets (or triplets)
that are [1] resolved; [2] unresolved in the specified tree.
\end{Value}
%
\begin{Section}{Functions}
\begin{itemize}

\item \code{ResolvedTriplets}: Counts the number of resolved/unresolved triplets

\end{itemize}
\end{Section}
%
\begin{Author}\relax
Martin R. Smith
\end{Author}
%
\begin{References}\relax
Brodal GS, Fagerberg R, Mailund T, Pedersen CNS, Sand A (2013).
``Efficient algorithms for computing the triplet and quartet distance between trees of arbitrary degree.''
\emph{SODA '13 Proceedings of the twenty-fourth annual ACM-SIAM symposium on Discrete algorithms}, 1814--1832.
doi:\nobreakspace{}\Rhref{http://doi.org/10.1137/1.9781611973105.130}{10.1137\slash{}1.9781611973105.130}.
\end{References}
\inputencoding{utf8}
\HeaderA{SharedQuartetStatus}{Quartet Status}{SharedQuartetStatus}
\aliasA{QuartetStatus}{SharedQuartetStatus}{QuartetStatus}
%
\begin{Description}\relax
Determines the number of quartets that are consistent within pairs of
cladograms.
\end{Description}
%
\begin{Usage}
\begin{verbatim}
SharedQuartetStatus(trees, cf = trees[[1]])

QuartetStatus(trees, cf = trees[[1]])
\end{verbatim}
\end{Usage}
%
\begin{Arguments}
\begin{ldescription}
\item[\code{trees}] A list of trees of class \code{\LinkA{phylo}{phylo}}, with identically-labelled tips.

\item[\code{cf}] Comparison tree of class \code{\LinkA{phylo}{phylo}}.  If unspecified,
each tree is compared to the first tree in \code{trees}.
\end{ldescription}
\end{Arguments}
%
\begin{Details}\relax
Given a list of trees, returns the number of quartet statements present in the
reference tree (the first tree in the list, if \code{cf} is not specified)
that are also present in each other tree.  A random pair of fully-resolved
trees is expected to share \code{choose(n\_tip, 4) / 3} quartets.

If trees do not bear the same number of tips, \code{SharedQuartetStatus} will
consider only the quartets that include taxa occurring in both trees.

From this information it is possible to calculate how many of all possible
quartets occur in one tree or the other, but there is not yet a function
calculating this; \Rhref{https://github.com/ms609/Quartet/issues/new}{let us know}
if you would appreciate this functionality.

The status of each quartet is calculated using the algorithms of
Brodal \emph{et al}. (2013) and Holt \emph{et al}. (2014), implemented in the
tqdist C library (Sand \emph{et al}. 2014).
\end{Details}
%
\begin{Value}
Returns a two dimensional array. Rows correspond to the
\begin{description}

\item[N] The total number of quartet \emph{statements} for two trees of \emph{n} tips,
i.e. 2 \emph{Q}.
\item[Q] The total number of quartets for \emph{n} tips.
\item[s] The number of quartets that are resolved identically in both trees.
\item[d] The number of quartets that are resolved differently in each tree.
\item[r1] The number of quartets that are resolved in tree 1, but not in tree 2.
\item[r2] The number of quartets that are resolved in tree 2, but not in tree 1.
\item[u] The number of quartets that are unresolved in both trees.

\end{description}

\end{Value}
%
\begin{Section}{Functions}
\begin{itemize}

\item \code{SharedQuartetStatus}: Reports split statistics obtained after removing all
tips that do not occur in both trees being compared.

\end{itemize}
\end{Section}
%
\begin{Author}\relax
Martin R. Smith
\end{Author}
%
\begin{References}\relax

Brodal GS, Fagerberg R, Mailund T, Pedersen CNS, Sand A (2013).
``Efficient algorithms for computing the triplet and quartet distance between trees of arbitrary degree.''
\emph{SODA '13 Proceedings of the twenty-fourth annual ACM-SIAM symposium on Discrete algorithms}, 1814--1832.
doi:\nobreakspace{}\Rhref{http://doi.org/10.1137/1.9781611973105.130}{10.1137\slash{}1.9781611973105.130}.

Estabrook GF, McMorris FR, Meacham CA (1985).
``Comparison of undirected phylogenetic trees based on subtrees of four evolutionary units.''
\emph{Systematic Zoology}, \bold{34}(2), 193--200.
doi:\nobreakspace{}\Rhref{http://doi.org/10.2307/2413326}{10.2307\slash{}2413326}.

Holt MK, Johansen J, Brodal GS (2014).
``On the scalability of computing triplet and quartet distances.''
In \emph{Proceedings of 16th Workshop on Algorithm Engineering and Experiments (ALENEX) Portland, Oregon, USA}.

Sand A, Holt MK, Johansen J, Brodal GS, Mailund T, Pedersen CNS (2014).
``tqDist: a library for computing the quartet and triplet distances between binary or general trees.''
\emph{Bioinformatics}, \bold{30}(14), 2079--2080.
ISSN 1460-2059, doi:\nobreakspace{}\Rhref{http://doi.org/10.1093/bioinformatics/btu157}{10.1093\slash{}bioinformatics\slash{}btu157}.

\end{References}
%
\begin{SeeAlso}\relax
\LinkA{SplitStatus}{SplitStatus}: Uses bipartition splits (groups/clades defined by
nodes or edges of the tree) instead of quartets as the unit of comparison.
\end{SeeAlso}
%
\begin{Examples}
\begin{ExampleCode}
 data('sq_trees')
 # Calculate the status of each quartet
 sq_status <- QuartetStatus(sq_trees)

 # Calculate Estabrook et al's similarity measures:
 SimilarityMetrics(sq_status)


\end{ExampleCode}
\end{Examples}
\inputencoding{utf8}
\HeaderA{SimilarityMetrics}{Tree Similarity Metrics}{SimilarityMetrics}
\aliasA{DoNotConflict}{SimilarityMetrics}{DoNotConflict}
\aliasA{ExplicitlyAgree}{SimilarityMetrics}{ExplicitlyAgree}
\aliasA{MarczewskiSteinhaus}{SimilarityMetrics}{MarczewskiSteinhaus}
\aliasA{QuartetDivergence}{SimilarityMetrics}{QuartetDivergence}
\aliasA{RobinsonFoulds}{SimilarityMetrics}{RobinsonFoulds}
\aliasA{SemiStrictJointAssertions}{SimilarityMetrics}{SemiStrictJointAssertions}
\aliasA{SteelPenny}{SimilarityMetrics}{SteelPenny}
\aliasA{StrictJointAssertions}{SimilarityMetrics}{StrictJointAssertions}
\aliasA{SymmetricDifference}{SimilarityMetrics}{SymmetricDifference}
%
\begin{Description}\relax
Functions to calculate tree similarity / difference metrics.
\end{Description}
%
\begin{Usage}
\begin{verbatim}
SimilarityMetrics(elementStatus, similarity = TRUE)

DoNotConflict(elementStatus, similarity = TRUE)

ExplicitlyAgree(elementStatus, similarity = TRUE)

StrictJointAssertions(elementStatus, similarity = TRUE)

SemiStrictJointAssertions(elementStatus, similarity = TRUE)

SymmetricDifference(elementStatus, similarity = TRUE)

RobinsonFoulds(elementStatus, similarity = FALSE)

MarczewskiSteinhaus(elementStatus, similarity = TRUE)

SteelPenny(elementStatus, similarity = TRUE)

QuartetDivergence(elementStatus, similarity = TRUE)
\end{verbatim}
\end{Usage}
%
\begin{Arguments}
\begin{ldescription}
\item[\code{elementStatus}] Two-dimensional integer array, with rows corresponding to
counts of matching quartets or partitions for each tree, and columns named
according to the output of \LinkA{QuartetStatus}{QuartetStatus} or \LinkA{SplitStatus}{SplitStatus}.

\item[\code{similarity}] Logical specifying whether to calculate the similarity
or dissimilarity.
\end{ldescription}
\end{Arguments}
%
\begin{Details}\relax
Estabrook \emph{et al.} (1985, table 2) define four similarity metrics in terms of the
total number of quartets (\emph{N}, their \emph{Q}), the number of quartets resolved in the same
manner in two trees (\emph{s}), the number resolved differently in both trees
(\emph{d}), the number resolved in tree 1 or 2 but unresolved in the other tree
(\emph{r1}, \emph{r2}), and the number that are unresolved in both trees (\emph{u}).

The similarity metrics are then given as below.  The dissimilarity metrics
are their complement (i.e. 1 - \emph{similarity}), and can be calculated
algebraically using the identity \emph{N} = \emph{s} + \emph{d} + \emph{r1} + \emph{r2} + \emph{u}.
\begin{itemize}

\item Do Not Conflict (DC): (\emph{s} + \emph{r1} + \emph{r2} + \emph{u}) / \emph{N}
\item Explicitly Agree (EA): \emph{s} / \emph{N}
\item Strict Joint Assertions (SJA): \emph{s} / (\emph{s} + \emph{d})
\item SemiStrict Joint Assertions (SSJA): \emph{s} / (\emph{s} + \emph{d} + \emph{u})

\end{itemize}


(The numerator of the Semistrict Joint Assertions similarity metric is given in
Estabrook \emph{et al}. (1985)'s table 2 as \emph{s} + \emph{d}, but this is understood, with
reference to the text to be a typographic error.)

Steel \& Penny (1993) propose a further metric, which they denote dQ,
which this package calculates using the function \code{SteelPenny}:
\begin{itemize}

\item Steel \& Penny's Quartet Metric (dQ): (\emph{s} + \emph{u}) / \emph{N}

\end{itemize}


Although defined using quartets, analagous values can be calculated using partitions
-- though for reasons listed elsewhere (see Smith 2019, supplementary text),
quartets offer a more meaningful measure of the amount of information
shared by two trees.

Another take on tree similarity is to consider the symmetric difference: that is,
the number of quartets or partitions present in one tree that do not appear in the
other, originally used to measure tree similarity by Robinson \& Foulds (1981).
\begin{itemize}

\item Robinson Foulds (RF): \emph{d1} + \emph{d2} + \emph{r1} + \emph{r2}

\end{itemize}


With quartets, \emph{d1} + \emph{d2} = 2 \emph{d}.

(Note that, given the familiarity of the Robinson Foulds distance metric, this
quantity is by default expressed as a difference rather than a similarity.)

To contextualize the symmetric difference, it may be normalized against:
\begin{itemize}

\item The total number of resolved quartets or partitions present in both trees (Day 1986):
\begin{itemize}

\item Day's Symmetric Difference (SD): (\emph{d1} + \emph{d2} + \emph{r1} + \emph{r2}) /
(\emph{d1} + \emph{d2} + 2 \emph{s} + \emph{r1} + \emph{r2})

\end{itemize}

\item The total distinctly resolved quartets or partitions (Day 1986):
\begin{itemize}

\item Marczewski-Steinhaus (MS): (\emph{d1} + \emph{d2} + \emph{r1} + \emph{r2}) /
(\emph{d1} + \emph{d2} + \emph{s} + \emph{r1} + \emph{r2})

\end{itemize}

\item The maximum number of quartets or partitions that could have been resolved, given the
number of tips (Smith 2019):
\item Quartet Divergence: (\emph{d1} + \emph{d2} + \emph{r1} + \emph{r2}) / 2 \emph{Q}

\end{itemize}


The partition equivalent to the latter will depend on the question being
asked, as Q should denote the maximum difference that \emph{could} have been
obtained.
\end{Details}
%
\begin{Value}
\code{SimilarityMetrics} returns a named two-dimensional array in which each row
corresponds to an input tree, and each column corresponds to one of the
listed measures.

\code{DoNotConflict} and others return a named vector describing the requested
similarity (or difference) between the trees.
\end{Value}
%
\begin{Author}\relax
Martin R. Smith
\end{Author}
%
\begin{References}\relax
Day WH (1986).
``Analysis of quartet dissimilarity measures between undirected phylogenetic trees.''
\emph{Systematic Biology}, \bold{35}(3), 325--333.
doi:\nobreakspace{}\Rhref{http://doi.org/10.1093/sysbio/35.3.325}{10.1093\slash{}sysbio\slash{}35.3.325}.

Estabrook GF, McMorris FR, Meacham CA (1985).
``Comparison of undirected phylogenetic trees based on subtrees of four evolutionary units.''
\emph{Systematic Zoology}, \bold{34}(2), 193--200.
doi:\nobreakspace{}\Rhref{http://doi.org/10.2307/2413326}{10.2307\slash{}2413326}.

Marczewski E, Steinhaus H (1958).
``On a certain distance of sets and the corresponding distance of functions.''
\emph{Colloquium Mathematicae}, \bold{6}(1), 319--327.
\url{https://eudml.org/doc/210378}.

Robinson DF, Foulds LR (1981).
``Comparison of phylogenetic trees.''
\emph{Mathematical Biosciences}, \bold{53}(1-2), 131--147.
ISSN 00255564, doi:\nobreakspace{}\Rhref{http://doi.org/10.1016/0025-5564(81)90043-2}{10.1016\slash{}0025\-5564(81)90043\-2}.

Smith MR (in review).
``Bayesian and parsimony approaches reconstruct informative trees from simulated morphological datasets.''
\emph{Biology Letters; preprint at BioRxiv}.
doi:\nobreakspace{}\Rhref{http://doi.org/10.1101/227942}{10.1101\slash{}227942}.

Steel MA, Penny D (1993).
``Distributions of tree comparison metrics---some new results.''
\emph{Systematic Biology}, \bold{42}(2), 126--141.
doi:\nobreakspace{}\Rhref{http://doi.org/10.1093/sysbio/42.2.126}{10.1093\slash{}sysbio\slash{}42.2.126}, \url{http://www.math.canterbury.ac.nz/{~}m.steel/Non{\_}UC/files/research/distributions.pdf}.

Smith MR (in review).
``Bayesian and parsimony approaches reconstruct informative trees from simulated morphological datasets.''
\emph{Biology Letters; preprint at BioRxiv}.
doi:\nobreakspace{}\Rhref{http://doi.org/10.1101/227942}{10.1101\slash{}227942}.
\end{References}
%
\begin{SeeAlso}\relax
\begin{itemize}

\item \LinkA{QuartetStatus}{QuartetStatus}: Calculate status of each quartet: the raw material
from which the Estabrook \emph{et al.} metrics are calculated.
\item \LinkA{SplitStatus}{SplitStatus}, \LinkA{CompareSplits}{CompareSplits}: equivalent metrics for bipartition splits.

\end{itemize}

\end{SeeAlso}
%
\begin{Examples}
\begin{ExampleCode}
  data('sq_trees')
  
  sq_status <- QuartetStatus(sq_trees)
  SimilarityMetrics(sq_status)
  QuartetDivergence(sq_status, similarity=FALSE)

\end{ExampleCode}
\end{Examples}
\inputencoding{utf8}
\HeaderA{sq\_trees}{Fifteen trees}{sq.Rul.trees}
\keyword{datasets}{sq\_trees}
%
\begin{Description}\relax
A list of class \code{multiPhylo} containing phylogenetic trees:
\begin{description}

\item[\code{ref\_tree}] A reference tree, bearing tips labelled 1 to 11.
\item[\code{move\_one\_near}] Tip 1 has been moved a short distance.
\item[\code{move\_one\_mid}] Tip 1 has been moved further.
\item[\code{move\_one\_far}] Tip 1 has been moved further still.
\item[\code{move\_two\_near}] Tips 10 \& 11 have been moved a short distance.
\item[\code{move\_two\_mid}] Tips 10 \& 11 have been moved further.
\item[\code{move\_two\_far}] Tips 10 \& 11 have been moved further still.
\item[\code{collapse\_one}] One node has been collapsed into a polytomy.
\item[\code{collapse\_some}] Several nodes have been collapsed.
\item[\code{m1mid\_col1}] Tree \code{move\_one\_mid} with one node collapsed.
\item[\code{m1mid\_colsome}] Tree \code{move\_one\_mid} with several nodes collapsed.
\item[\code{m2mid\_col1}] Tree \code{move\_two\_mid} with one node collapsed.
\item[\code{m2mid\_colsome}] Tree \code{move\_two\_mid} with several nodes collapsed.
\item[\code{opposite\_tree}] A tree that is more different from \code{ref\_tree} than expected by chance.
\item[\code{random\_tree}] A random tree.

\end{description}

\end{Description}
%
\begin{Usage}
\begin{verbatim}
sq_trees
\end{verbatim}
\end{Usage}
%
\begin{Format}
An object of class \code{multiPhylo} of length 15.
\end{Format}
\inputencoding{utf8}
\HeaderA{SymmetricDifferenceLineEnds}{Plot lines of equal Symmetric Difference on a ternary plot}{SymmetricDifferenceLineEnds}
\aliasA{SymmetricDifferenceLines}{SymmetricDifferenceLineEnds}{SymmetricDifferenceLines}
%
\begin{Description}\relax
Assumes that tree 1 is perfectly resolved, but that the resolution
of tree 2 can vary.
\end{Description}
%
\begin{Usage}
\begin{verbatim}
SymmetricDifferenceLineEnds(nsd)

SymmetricDifferenceLines(nsd, ...)
\end{verbatim}
\end{Usage}
%
\begin{Arguments}
\begin{ldescription}
\item[\code{nsd}] Vector specifying normalized symmetric differences to plot

\item[\code{...}] Further parameters to pass to
\code{\LinkA{TernaryLines}{TernaryLines}}
\end{ldescription}
\end{Arguments}
%
\begin{Value}
Returns a matrix of dim \code{(length(nsd), 6)``, with columns named }r2a\code{, }da\code{, }sa\code{, }r2b\code{, }db\code{ans}sb\code{.  Lines from }a\code{to}b` in each row connect points
of equal symmetric difference.
\end{Value}
%
\begin{Section}{Functions}
\begin{itemize}

\item \code{SymmetricDifferenceLines}: Plot the lines onto the active ternary plot

\end{itemize}
\end{Section}
%
\begin{Author}\relax
Martin R. Smith
\end{Author}
\inputencoding{utf8}
\HeaderA{TQDist}{tqDist wrapper}{TQDist}
\aliasA{ManyToManyQuartetAgreement}{TQDist}{ManyToManyQuartetAgreement}
\aliasA{SingleTreeQuartetAgreement}{TQDist}{SingleTreeQuartetAgreement}
\aliasA{TQAE}{TQDist}{TQAE}
%
\begin{Description}\relax
Convenience function that takes a list of trees, writes them to the text file
expected by the C implementation of tqDist (Sand \emph{et al.} 2014).
tqDist is then called, and the temporary file is deleted when analysis is complete.
\end{Description}
%
\begin{Usage}
\begin{verbatim}
TQDist(treeList)

TQAE(treeList)

ManyToManyQuartetAgreement(treeList)

SingleTreeQuartetAgreement(treeList, comparison)
\end{verbatim}
\end{Usage}
%
\begin{Arguments}
\begin{ldescription}
\item[\code{treeList}] List of phylogenetic trees, of class \code{list} or
\code{phylo}. All trees must be bifurcating.

\item[\code{comparison}] A single tree against which to compare the trees in treeList
\end{ldescription}
\end{Arguments}
%
\begin{Details}\relax
Quartets can be resolved in one of five ways, which
Brodal \emph{et al}. (2013) and Holt \emph{et al}. (2014) distinguish using the letters
A--E, and Estabrook (1985) refers to as:
\begin{itemize}

\item A: \emph{s} = resolved the \strong{s}ame in both trees
\item B: \emph{d} = resolved \strong{d}ifferently in both trees
\item C: \emph{r1} = \strong{r}esolved only in tree \strong{1}
\item D: \emph{r2} = \strong{r}esolved only in tree \strong{2} (the comparison tree)
\item E: \emph{u} = \strong{u}nresolved in both trees

\end{itemize}

\end{Details}
%
\begin{Value}
\code{TQDist} returns the quartet distance between each pair of trees

\code{TQAE} returns the number of resolved quartets in agreement between
each pair of trees (A in Brodal \emph{et al}. 2013) and the number of quartets
that are unresolved in both trees (E in Brodal \emph{et al}. 2013).

\code{ManyToManyQuartetAgreement} returns a three-dimensional array listing,
for each pair of trees in turn, the number of quartets in each category.

\code{SingleTreeQuartetAgreement} returns a two-dimensional array listing,
for tree in \code{treeList}, the total number of quartets and the
number of quartets in each category.
The \code{comparison} tree is treated as \code{tree2}.
\end{Value}
%
\begin{Section}{Functions}
\begin{itemize}

\item \code{TQAE}: Number of agreeing quartets that are resolved / unresolved

\item \code{ManyToManyQuartetAgreement}: Agreement of each quartet, comparing each pair of trees in a list

\item \code{SingleTreeQuartetAgreement}: Agreement of each quartet in trees in a list with the
quartets in a comparison tree

\end{itemize}
\end{Section}
%
\begin{Author}\relax
Martin R. Smith
\end{Author}
%
\begin{References}\relax
Brodal GS, Fagerberg R, Mailund T, Pedersen CNS, Sand A (2013).
``Efficient algorithms for computing the triplet and quartet distance between trees of arbitrary degree.''
\emph{SODA '13 Proceedings of the twenty-fourth annual ACM-SIAM symposium on Discrete algorithms}, 1814--1832.
doi:\nobreakspace{}\Rhref{http://doi.org/10.1137/1.9781611973105.130}{10.1137\slash{}1.9781611973105.130}.

Estabrook GF, McMorris FR, Meacham CA (1985).
``Comparison of undirected phylogenetic trees based on subtrees of four evolutionary units.''
\emph{Systematic Zoology}, \bold{34}(2), 193--200.
doi:\nobreakspace{}\Rhref{http://doi.org/10.2307/2413326}{10.2307\slash{}2413326}.

Holt MK, Johansen J, Brodal GS (2014).
``On the scalability of computing triplet and quartet distances.''
In \emph{Proceedings of 16th Workshop on Algorithm Engineering and Experiments (ALENEX) Portland, Oregon, USA}.

Sand A, Holt MK, Johansen J, Brodal GS, Mailund T, Pedersen CNS (2014).
``tqDist: a library for computing the quartet and triplet distances between binary or general trees.''
\emph{Bioinformatics}, \bold{30}(14), 2079--2080.
ISSN 1460-2059, doi:\nobreakspace{}\Rhref{http://doi.org/10.1093/bioinformatics/btu157}{10.1093\slash{}bioinformatics\slash{}btu157}.
\end{References}
\printindex{}
\end{document}
